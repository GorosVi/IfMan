\documentclass[a4paper,12pt]{report}
\usepackage{GS6}

\begin{document}
	\sloppy

	\def \LineB {Конспект по дисциплине}
	\def \LineC {“Основы современного управления”}
	\def \LineD {\vspace{2cm}}
	\def \nocredits {}

\maketitle

\subsection*{Содержание}
\maketoc





%===================================== Лекция - 02.09.14 ===========================================

\section{Основы управленческой деятельности}


\subsection{Управленческие революции}

	\begin{enumerate}
		\item	Религиозно-коммерческая
		\item	Административная - Хаммурапи
		\item	$\sim$
	\end{enumerate}



\subsection{Классификация организаций}

	По отношению к прибыли
	\begin{itemize}
		\item	Коммерческие
		\item	Некоммерческие
	\end{itemize}

	\begin{itemize}
		\item	Формальные - официально зарегистрированная организация, имеющая установленную структуру
			должностей, принятые в ней нормы и правила поведения.
		\item	Неформальные - нигде официально не зарегистрированы, целью имеют собрание по интересам.
	\end{itemize}

	По форме собственности
	\begin{itemize}
		\item Государственные
		\item Частные
		\item Муниципальные - владелец - местные органы власти
	\end{itemize}

	По характеру потребности клиентов
	\begin{itemize}
		\item	Массовое производство
		\item	Серийное - мелко/средне/крупносерийное
		\item	Единичное - нестабильные условия труда, редкие товары
	\end{itemize}

	КЗО - коэффициент закрепления операций

	Количество операций/Количество рабочих мест

\subsubsection{Формирование отношений внутри организации}
	\begin{itemize}
		\item	Корпоративные
		\item	Индивидуалистские
		\item	Эдхократические
		\item	Партисипативные
	\end{itemize}






%===================================== Лекция - 09.09.14 ===========================================
\subsection{Жизненный цикл организации}

	... Lost ...

	\begin{verbatim}
	     ________
	    /        \
	   /          \
	 _/            \
	/               \
	\end{verbatim}


\subsubsection{Основные признаки упадка компании}

	\begin{enumerate}
		\item	Общий спад спроса на продукцию
		\item	Возрастает конкурентная сила поставщиков - большая зависимость от них
		\item 	Возрастает роль цены и качества в конкурентной борьбе
		\item	Возрастает сложность управления приростом производственных мощностей
		\item	Усложняетса процесс внедрения товарных инноваций
		\item	Как итог, снижается прибыльность
	\end{enumerate}

	Выручка - все полученные в результате реализации продукции/услуг.

	Затраты - расходы, понесенные предприятием

	Прибыль - Выручка - Затраты

	Чистая прибыль - прибыль за вычетом налогов и сборов


\subsection{Внешняя и внутренняя среда организации}

	Внешняя среда организации:
	\begin{itemize}
		\item	Прямого воздействия: поставщики (сырьё, материалы, энергия; капитала и фин. услуг, банки,
			трудовых ресурсов); потребители; конкуренты; законы и гос.органы
		\item	Косвенного воздействия: социально-культурные факторы (демографическое состояние общества,
			отношения с потребителями); Политическая ситуация в стране и мире; Экономическая ситуация в
			стране и мире; Уровень научно-технического прогресса.
	\end{itemize}

	Характеристики внешней среды
	\begin{itemize}
		\item	Взаимодействие с внешней средой
		\item	Сложность внешней среды
		\item	Подвижность - скорость изменения отрасли
		\item	Неопределённость внешней среды - относительное количество информации о её состоянии
	\end{itemize}

	Методы диагностики внешней среды:
	\begin{itemize}
		\item	SWOT-анализ (анализ сильных/слабых сторон, угроз) - изучает положение бизнес-единицы 
			на рынке относительно конкурентов.
	% Strenghts  | Opportunities
	% Weakencies | Threats
		\item	PEST-анализ (Political/Economical/Society/Technological factors) - отличие от SWOT в 
			изучении рынка
	\end{itemize}


\subsubsection{Внутренняя среда организации}
	Основные факторы
	\begin{itemize}
		\item	Цели организации
		\item	Структура организации
		\item	Ресурсы
		\item	Организационная культура
	\end{itemize}





%===================================== Лекция - 07.10.14 ===========================================
\subsection{Теории мотивации}

	Герцберг:
	Процесс приобретения удовлетворенности и нарастания неудовлетворённости  считаются разными процессам

	Факторы, влияющие на мотивацию:
	\begin{enumerate}
		\item	Мотивирующие факторы: Признание, продвижение, возможности роста.
		\item	Факторы здоровья (потребности в устранении трудностей) - заработная плата, условия 
			рабочего места, распорядок и режим, взаимоотношения с сотрудниками и руководством.
	\end{enumerate}

	Герцберг доказал, что отсутствие первой группы факторов (мотив.) не приведёт к большой неудовлетворённости,
	 а их наличие повысит качество выполняемой работы. Отсутствие же второй группы факторов приводит к
	 неудовлетворенности работника, а их наличие - не факт его удовлетворённости.


	Теории ожидания В. Врума: поведение человека всегда связано с выбором наиболее привлекательной альтернативы 
	 из имеющихся. Человек ведёт себя в соответствии с тем, что по его мнению произойдёт, если он предпримет
	 определённые усилия.
	
	Врум выделяет 2 группы ожиданий:
	\begin{itemize}
		\item 	Ожидания, связывающие усилия и отдых от работы.
		\item	Ожидания, связывающие выполнение работы и полученные результаты.
	\end{itemize}

	Усилия => Результат => Вознаграждение
	Врум вводит понятие валентности - ожидаемая ценность вознаграждения.


	Теория постановки целей.
	Согласно теории, поведение человека определяется целями, поставленными им самим, либо кем-либо.
	Качество выполнения работы зависит от двух факкторd: организационные и способностей работника.



	Стейси Адамс - теория равенства.
	Говорит о справедливости соотношения результатов работы и вознаграждения.

	Концепция партисипативного управления.
	Если работнику предоставлена возожность участвовать в процессе управления она повышает отдачу при выполнении
	 работы. Исходит из того, что человек всегда стремится участвовать в организационных процессах, и если ему
	 предоставляется такая возможность, то он начинает работать с большей отдачей.




\subsection{Организационные структуры управления}

	Организационная структура управления - форма разделения и кооперации управленческой деятельности, в рамках 
	 которой  осуществляется управление организацией.
	\begin{itemize}

	\item	Функциональная структура управления: В чистом виде не встречается на предприятиях. Процесс деление
		 организации на отдельные элементы, каждый из которых имеет свою чётко определённую задачу. Такая ОСУ
		 направлена на стимулирование труда работников, а также экономию, обусловленную ростом масштабов
		 производства. Недостатки - трудности в координировании деятельности деятельности и в календарном
		 планировании. Такая ОСУ не подходит для организаций с широкой номенклатурой продукции, предприятий,
		 работающих в быстроменяющейся средой, и действующих в международном масштабе.

	\begin{verbatim}
	        Высший руководитель
	                     |
	        [Функциональные руководители]
	                /        \     /    \
	        <Исполнители><Исполнители><Исполнители>
	\end{verbatim}


 	\item 	Линейная ОСУ встречается на небольших предприятиях, выпускающих однородную продукцию по несложной
 		 технологии. Минус - при росте численности требует переустройства.

		\begin{verbatim}
	               Высший руководитель
	                     |
	              [Линейные руководители]
	              |         |           |
	        <Исполнители><Исполнители><Исполнители>
		\end{verbatim}

	\item 	Линейно - функциональная

		\begin{verbatim}
	             Высший руководитель
	                     |
	           [Функциональные руководители]
	              /        \     /    \
	             [Линейные руководители]
	              |         |           |
	        <Исполнители><Исполнители><Исполнители>
		\end{verbatim}

	\item	Матричная - аннулирует принцип единоначалия - характерны для проектных организаций.

		\begin{ztable}{rcc}
			& Менеджер 1 & Менеджер 2\\
		Исполнитель 1&	X    &     X     \\
		Исполнитель 1&	X    &           \\
		Исполнитель N&	     &     X     \\
		\end{ztable}

	\item 	Дивизиональная ОСУ - на основе подразделений, осуществляющих полный цикл производственно - хозяйственной
		 деятельности. Варианты разделения: продуктовый признак; территориальный признак.
	\end{itemize}


%===================================== Лекция - 21.10.14 ===========================================
	\subsection{Стратегии развития}
	Лица, принимающие решения:

	Индивидуалистские стратегии
	\begin{itemize}
		\item Пессимистическая
		\item Оптимистическая
		\item Рациональная
	\end{itemize}

		Групповые стратегии (более 6-8 человек)
	\begin{itemize}
		\item Принцип большинства голосов
		\item Принцип диктатора (единоличное принятие решений)
		\item Принцип Курно (от нескольких коалиций по 1 чел - согласованность до невыгодности изменения решения)
		\item Принцип Паретто (всем невыгодно менять принимаемое решение)
	\end{itemize}

\subsubsection{Методы принятия решений}


	Методы принятия решений делятся на формализованные (математические) и неформализованные

	Формализованные методы основаны на получении количественных результатов вычислений.
	Они могут быть
	\begin{itemize}
		\item Аналитические (между прогнозом и результатом установлены формализованные зависимости)
		\item Статистические методы (При моделировании ситуации применяются статистические методы (не
			только фактические, но и результаты моделирования))
		\item Теоретико-игровые методы (путём моделирования в теории игр прогнозируются варианты развития
			(например в случае злонамеренной конкуренции))
	\end{itemize}

	Формализованные методы используют экономико-математические модели и методы для обоснования и выбора 
	 оптимального решения. (напр. "метод экспертных оценок")


	Неформализованные методы
	\begin{itemize}
		\item Метод "мозгового штурма" (оптимальный размер группы 6-8 чел. Есть шанс получить наибольший охват
			 приложений)
		\item Метод "Дельфы" (есть несколько участников группы, изолированных, каждый предлагает вариант решения.
			 Участники выбирают лучшее решение, до полного консенсуса)
		\item Метод сценариев
		\item Метод дерева решений (строится граф решений и последствий)
	\end{itemize}

	\subsubsection{Методы субъективных измерений предпочтительности решения}
	\begin{itemize}
		\item Ранжирование (упорядочение объектов в зависимости от их предпочтительности по порядковой шкале.
			 Проводится на основе некоторого заранее известного критерия)
		\item Метод "Парное сравнение" (сравнивают объекты между собой)\\
		Матрица сравнений:

		\begin{ztable}{Lccc}
		&X1 &X2 &x3 \\
		X1 &= &> &> \\
		X2 &< &= &< \\
		X3 &> &< &= \\
		\end{ztable}

		\item Непосредственная оценка (на каждый объект накладывается определённая шкала оценки, аки БАРС)
		\item Последовательное сравнение (На первом этапе объекты ранжируются. Далее оценивается каждый объект,
			 и при решении предпочтительности, чем все остальные ему присваивается наибольшая оценка)
	\end{itemize}


	[Консультации - Нечётные ПН, $12^{00}-12^{40}$ - каф.менедж, во флигеле 4-этажном, на 3-м этаже]

	[4 ноября - пришлют теорматериал, лекций не будет - upd [G] - материал так и не прислали, считаем утерянным] 
	

%===================================== Лекция - 11.11.14 ===========================================
\subsection{Коммуникации}	
	
	Элементы
	\begin{itemize}
	\item	Источник (отправитель)		
	\item	Канал - средство передачи информации
	\item	Получатель - расшифровка и интерпретация принятой информации
	\end{itemize}		
	
	
	Этапы:
	\begin{itemize}
	\item	Зарождение 
	\item	Кодирование
	\item	Передача сигнала
	\item	Расшифровка - приём
	\item	Обратная связь - опционально
	\end{itemize}		

	Основные помехи и барьеры для эффективной коммуникации:
	\begin{itemize}
	\item	Отвлечения - пропуски в повествовании, пропуски принимающего
	\item	Неправильная интерпретация
	\item	Статусные различия между начальником и подчинёнными, недопонимание ситуации
	\item	Различие во взглядах, ценностном восприятии
	\end{itemize}

	\subsubsection{Типы коммуникаций}
	
	\begin{itemize}
	\item	Внутриличностная коммуникация
	
	\item	Межличностные коммуникации
	
	\item	Коммуникация в малых группах (10-12 чел.)
	
	\item	Общественные коммуникации - отсутствие обратной связи
	
	\item	Внутренняя оперативная коммуникация - бюллетени, летучки.
	
	\item	Внешняя оперативная коммуникация - переписка с поставщиками, итп.
	
	\item	Личностная коммуникация - случайный обмен информацией, без цели.
	\end{itemize}

	Организационные факторы, влияющие на коммуникацию:
	\begin{itemize}
	\item	Должностное положение
	\item	Стиль управления (авторитарный, демократический, либеральный)
	\item	Разделение труда
	\end{itemize}

	\subsubsection{Соввершенствование коммуникаций в организации}
	\begin{itemize}
	\item	Регулирование информационных потоков - распределение информации сотрудникам.
	\item	Системы сбора предложений (различные бюллетени)
	
	\end{itemize}

	\subsection{Организационная культура}
	
	Организационная культура - набор наиболее важных предположений,принимаемых членами организации и получающих
	 выражениях в заявляемых организацией ценностях, задающих людям ориентиры их поведения и действий.
	
	Содержание организационной культуры
	\begin{itemize}
	\item	Осознание себя и своего места в организации
	\item	Коммуникационная система и язык общения
	\item	Внешний вид, одежда и представление себя на работе
	\item	Осознание времени, отношение к нему и его использование
	\item	Взаимоотношения между людьми
	\item	Ценности и нормы - ориентиры в нравственной и поведенческой плоскости
	\item	Процесс развития работника и его обучение
	\item	Трудовая этика и мотивирование
	\item	Вера во что-либо или расположение к чему-то
	\item	Привычки, бытовые традиции в данной области (питание итп) 
	\end{itemize}
	
	\subsubsection{Формирование организационной культуры}
	ФОК зависит от деловой среды в целом и отрасли в частности, образцов национальной культуры.
	
	Процесс внешней адаптации характеризует поиск организацией своего места на рынке.
	Процесс внутренней интеграции

	\subsubsection{Поддержание организационной культуры}
	
	Текучка кадров приводит к появлению в компании новых сотрудников со своими представлениями об организационной
	 культуре, могущими вступить в противоречие с исходными положениями в организации
	
	Сила организационной культуры определяется 3 моментами:
		Степенью разделяемости членами организации
		Ясностью приоритетов культуры
		"Толщина" организационной культуры - количество важных предположений, разделяемых работниками.
		

% next lecture starts here





\end{document}
